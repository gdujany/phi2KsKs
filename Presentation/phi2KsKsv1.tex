\documentclass{beamer}
\usepackage{beamerthemesplit,graphicx,wrapfig, slashed, subfigure,verbatim,hyperref,color,colortbl,tabularx}
% \usepackage{graphicx,wrapfig, slashed, subfigure,verbatim,hyperref,color,colortbl,tabularx}
\usepackage{amsmath,amssymb}
\usepackage{helvet} %font helvetica
\usepackage{siunitx}
\usepackage{pdftricks}
\usepackage{pstricks}
\usepackage{beamerfoils}
\usepackage{multicol}
%\usepackage{babel}
\usepackage{feynmp-auto}

\usepackage[utf8]{inputenc}
\usepackage[T1]{fontenc}
\usepackage{ulem}
%\usepackage{epstopdf}
\usepackage{color, colortbl}
%\usepackage{biblatex}

% \usetheme{Hannover}
\useoutertheme{infolines2}
%\font\ttfAkkuratLightOffice AkkuratLightOffice-Regular at10pt
\colorlet{structure}{green!50!black}
\definecolor{tugreen}{RGB}{132,184,24}
\definecolor{tugrey}{RGB}{178,179,182}
\definecolor{tured}{RGB}{205,0,47}
\setbeamercolor{palette primary}{bg=tugreen,fg=black}
\setbeamercolor{palette secondary}{bg=tugrey!50!tugreen,fg=black}
\setbeamercolor{palette quaternary}{fg=black, bg=tugrey}
\setbeamercolor{caption name}{fg=tugreen}
\setbeamercolor{palette tertiary}{fg=black,bg=tugrey}


\setbeamercolor{palette compare}{bg=white!80!tugreen,fg=black}
\setbeamercolor{palette misc}{bg=white!80!tugreen,fg=black}
\setbeamercolor{palette white}{bg=white!99!black,fg=black}

\setbeamercolor{itemize item}{fg=tugreen}
\setbeamercolor{itemize subitem}{fg=tugreen}
\setbeamercolor{itemize subsubitem}{fg=tugreen}
\setbeamercolor{enumerate item}{fg=tugreen}
\setbeamertemplate{itemize item}[square]
\setbeamercolor{titlelike}{fg=tugreen, bg=white}
\setbeamertemplate{navigation symbols}{}

\title{Measurement of CPT Invariance in $\phi\rightarrow K_sK_s$}
\subtitle{A comparison of two approaches}

\author{\underline{Sonja Bartkowski}, Lorenzo Capriotti, Giulio Dujany, \\Jonathan Harrison,  \& George Lafferty}
%\institute[TU Dortmund]{
%\scriptsize Technische Universität Dortmund\\
%Lehrstuhl für Experimentelle Physik IV \\ 
%}
\date{August 17, 2015\\ \vspace{0.5cm}
\vspace*{1cm}
\begin{columns}
        \column{.2\textwidth}  \centering \includegraphics[width=0.75\textwidth]{TitlePage/AN_cernlogo.pdf}
	\column{.22\textwidth} \centering \includegraphics[width=0.8\textwidth]{lhcb_logo.pdf}
	\column{.25\textwidth} \centering \includegraphics[width=\textwidth]{UniOfManchesterLogo.pdf}
	        \column{.3\textwidth} \centering \includegraphics[width=\textwidth]{TitlePage/tud_logo_cmyk.pdf}
\end{columns}}
\MyLogo{\includegraphics[width=2.5cm]{lhcb_logo.pdf}} 

\begin{document}

\LogoOff
\frame{\titlepage}

\begin{frame}[fragile]
\frametitle{Introduction \& Terminology}
\vspace*{-.2cm}
\begin{beamercolorbox}[rounded=true,shadow=true]{palette misc}
\begin{itemize}
\item Want to measure CPT invariance
\item System of choice: $\phi \rightarrow K_S K_L$
\begin{itemize}
\item CPTV $\Rightarrow$ loss of coherence
\end{itemize}
\item Two possible approaches:
\begin{itemize}
\item inclusive $\phi$ production
\item $\phi$ from decays $D_s \rightarrow \phi \pi$
\end{itemize}
\item Comparison of both approaches
\end{itemize}
\end{beamercolorbox}
\begin{beamercolorbox}[rounded=true,shadow=true]{palette misc}
\textbf{Signal:} $\phi \rightarrow K_S K_S \rightarrow \pi^+\pi^-\pi^+\pi^-$\\
\textbf{SM background: }resulting from CPV, $\phi \rightarrow K_L K_S \rightarrow \pi^+\pi^-\pi^+\pi^-$\\
\textbf{Regeneration background: } regeneration $K_L \rightarrow K_S$ in material\\
\textbf{Combinatoric background} Prompt kaons and pions
\end{beamercolorbox}
\end{frame}




\LogoOff
\begin{frame}[fragile]
\frametitle{Selection - inclusive $\phi$}

\begin{beamercolorbox}[rounded=true,shadow=true]{palette misc}
\centering{Inclusive $\phi$ production}
\begin{itemize}
\item Stripping PhiToKSKS\_PhiToKsKsLine
\begin{itemize}
\item[$\pi$] 
TRGHOSTPROB < 0.35\\
P $>$ 2.GeV\\
MIPCHI2DV(PRIMARY) $>$ 9.\\
\item[$K_S$]ADMASS('KS0') < 35.MeV\\
VFASPF(VCHI2) < 25.\\
\item[$\phi$]LL or LD combinations $^{*)}$\\
APT > 400 MeV\\
VFASPF(VCHI2/VDOF) < 6\\
MIPCHI2DV(PRIMARY) < 9\\
M < 1100 MeV\\
\end{itemize}
\item $\SI{1010}{MeV}<$phi\_M$<\SI{1030}{MeV}$
\end{itemize}
\end{beamercolorbox}
$^{*)}$ because of regeneration, KLOE follows the same approach



\end{frame}
\LogoOn




\LogoOff
\begin{frame}[fragile]
\frametitle{Selection - $D_s \rightarrow \phi \pi$}
\vspace*{-2mm}
\begin{beamercolorbox}[rounded=true,shadow=true]{palette misc}
\vspace*{-3mm}
\begin{itemize}
\item Selection (inspired by PhiToKSKS\_PhiToKsKsLine and other charm lines) on CHARMCOMPLETEEVENT.DST

\begin{columns}
\begin{column}{.01\textwidth}
$\frac{}{}$
\end{column}
\begin{column}{.46\textwidth}
\vspace*{-.3mm}
\begin{itemize}
\item[$\pi$($K_S$)]
PT $>$ 150 MeV\\
BPVIPCHI2() $>$ 1.0\\
TRCHI2DOF $<$ 5\\
TRGHOSTPROB $<$ 0.3\\
\item[$K_S$] ADMASS('KS0') $<$ 35 MeV\\
VFASPF(VCHI2) $<$ 2.\\
PT $>$ 200 MeV\\
BPVVD $>$ 10.0 mm\\
BPVVDCHI2 $>$ 100\\
VFASPF(VCHI2PDOF) $<$ 10\\
BPVDIRA $>$ 0.999\\
\end{itemize}


\end{column}
\begin{column}{.48\textwidth}
\begin{itemize}
\item[$\phi$]LL or LD combinations\\
ADMASS('phi(1020)')$<$70 MeV\\
VFASPF(VCHI2/VDOF) $<$ 6\\
APT $>$ 400 MeV\\
\item[$\pi$($D_S$)]TRGHOSTPROB $<$ 0.35\\
P $>$ 2 GeV\\
MIPCHI2DV(PRIMARY) $>$ 9 \\
\item[$D_S$]ADMASS('D\_s+') $<$ 150MeV\\
(BPVVDCHI2 $>$ 16.0) or (BPVLTIME() $>$ 0.150 ps)\\
VFASPF(VCHI2/VDOF) $<$ 25.0\\
\end{itemize}

\end{column}
\end{columns}

\item \SI{1010}{MeV}$<$phi\_M$<$\SI{1030}{MeV} \& \SI{1955}{MeV}$<$Ds\_M$<$\SI{1985}{MeV}
\item IPCHI2 $\geq$ 15,\small{ (possible to tighten cut if more MC statistics available)}
\end{itemize}
\end{beamercolorbox}




\end{frame}
\LogoOn


\LogoOff
\begin{frame}
\frametitle{Efficiencies}
\begin{scriptsize}

\resizebox{\textwidth}{!}{
\begin{tabular}{ccc}
&Inclusive $\phi$ & $D_s \rightarrow \phi \pi$ \\
\hline 
\hline
Cross section (\SI{14}{TeV})\\
LHCb acceptance & \SI{ 3516 }{\micro\barn} & \SI{ 388 }{\micro\barn} \\
\hline
Branching fractions & 34.2 \% & 4.5  \% $\cdot$  34.2 \% \\
\hline
Fiducial cuts efficiency & 2.5 \% & 7.0 \%\\
\hline
Prob. $K_sK_s \rightarrow 4\pi$, \\
 exactly 1 (2) decays inside bp &\multicolumn{2}{c}{$ 15.1 \% $($ 2.8 \% $)} \\
\hline
Prob. $K_sK_L \rightarrow 4\pi$ (CPV),\\
exactly 1 (2) decays inside bp &\multicolumn{2}{c}{$ 3.98\cdot 10^{-7} $($ 4.99\cdot 10^{-10} $)} \\
\hline
Upper limit KLOE prob. $K_sK_L \rightarrow 4\pi$ (CPV \\
+ CPTV), exactly 1 (2) decays inside bp &\multicolumn{2}{c}{$  5.13\cdot 10^{-7} $($ 1.64\cdot 10^{-8} $)}\\
\hline
Reconstruction \& selection efficiency &  7.9 \%( 7.6 \%)  & 1.4 \%( 4.2 \%)\\
\hline
L0 efficiency&  16.1 \%( 18.6 \%)&  22.4 \%( 18.2 \%)\\
HLT1 efficiency&  13.7 \%( 16.7 \%)& 45.5 \%( 25.0 \%)\\
HLT2 efficiency&  65.6 \%( 100.0 \%)& 75.0 \%( 100.0 \%)\\
\hline
\hline
Total efficiency SM background &$ 4.39\cdot 10^{-5} $($ 5.85\cdot 10^{-5} $)&$ 1.02\cdot 10^{-4} $($ 1.32\cdot 10^{-4} $)\\
Expected events SM background / fb$^{-1}$ &$21 $($ 3.51\cdot 10^{-2} $)&$ 2.43\cdot 10^{-1} $($ 3.94\cdot 10^{-4} $)\\
Upper limit for signal (KLOE) &$ 27 $($ 1.15 $)&$ 3.13\cdot 10^{-1} $($ 1.29\cdot 10^{-2} $)\\
Background (data 2012) / fb$^{-1}$ & 163110 ( 29120 ) &  1170 ( 6100 )\\
\end{tabular}
}

\end{scriptsize}



\vspace*{2cm}

Background retention (estimated from minimum bias MC) $\sim 10^{-7}$
\end{frame}
\LogoOn






\LogoOff
\begin{frame}
\frametitle{Mass plots - inclusive $\phi$}
\begin{center}
\includegraphics[width = .90\textwidth]{incl_phi_M.pdf}
\end{center}

\end{frame}
\LogoOn

\LogoOff
\begin{frame}
\frametitle{Mass plots - $D_S\rightarrow \phi\pi$}
\begin{center}
\includegraphics[width = .90\textwidth]{Ds_M.pdf}
\end{center}
\end{frame}
\LogoOn

\LogoOff
\begin{frame}
\frametitle{Mass plots - $D_S\rightarrow \phi\pi$}
\begin{center}
\includegraphics[width = .90\textwidth]{Ds_phi_M.pdf}
\end{center}

\end{frame}
\LogoOn



\LogoOff
\begin{frame}
\frametitle{Comparisons }
\vspace*{-.4cm}
Good agreement with other studies!
\begin{beamercolorbox}[rounded=true,shadow=true]{palette misc}
\begin{center}
$D_0 \rightarrow K_sK_s$ by Markward Britsch\\
\vspace*{.5cm}
\scriptsize
\begin{tabular}{c|c|c}
 & M. Britsch & inclusive/$D_s$ approach \\ 
\hline 
Reconstruction \& selection  & 0.2-0.5\% & 0.2\%/ 0.1\% \\ 
\hline 
Efficiency L0+Hlt1 & 3\% & 10\% / 4\%\\ 
\hline 
Efficiency L0+Hlt1+Hlt2 & 1\% & 7\% / 4\%\\ 
\end{tabular} 

\end{center}


\end{beamercolorbox}
\begin{beamercolorbox}[rounded=true,shadow=true]{palette misc}
\begin{center}
$X\rightarrow K_sK_s$ by Thomas Ruf\\
\vspace*{.5cm}
\scriptsize
\begin{tabular}{c|c|c}
 & T. Ruf & inclusive/$D_s$ approach \\ 
\hline 
Acceptance \& reconstruction & 10\% &  \\ 
Reconstruction \& selection  & & 0.2\%/ 0.1\% \\ 
\hline 
Probability of at least 1 $K_s$ decaying in beampipe & 50\% & 44\% / 42\%\\ 
\end{tabular} 
\end{center}


\end{beamercolorbox}

\end{frame}
\LogoOn

\LogoOff
\begin{frame}[fragile]
\frametitle{Backgrounds}
Estimates from minimum bias MC (42 M events). Number in brackets is the the number of background events with physical $K_s$
\begin{center}
\begin{tabular}{c|c|c}
Background category & inclusive $\phi$ & $D_s \rightarrow \phi \pi$ \\ 
\hline 
light flavour & 17(17) & 0 \\ 
$b\overline{b}$ & 1(1) & 0 \\ 
different PV & 3(2) & 0 \\ 
physical bkg, partl. reconstructed & 1(1) & 1(1) \\ 
ghosts & 0 & 1(0) \\ 
\hline 
total & 21(20) & 2(1) \\  
\end{tabular} 
\end{center}
Remaining background for inclusive $\phi$ mostly irreducible
\end{frame}



\begin{frame}
\frametitle{Time Resolution}
\vspace*{-4mm}
\begin{center}
\includegraphics[width=.85\textwidth]{/home/soonja/lxplus/phi2KsKs/phi2KsKs/preliminaryStudies/timeResolution.pdf}
\end{center}
\vspace*{-4mm}
Resolution of a few ps; the core that has a resolution of less than 1 ps
\end{frame}
\LogoOn




\LogoOff
\begin{frame}[fragile]
\frametitle{Summary}

\begin{beamercolorbox}[rounded=true,shadow=true]{palette misc}

\begin{itemize}
\item Selection for $D_S \rightarrow \phi \pi$ was implemented
\item Compared the inclusive and $D_S$ strategies
\item For both approaches, the background dominates
\item In 2012 data, there is no $\phi$ peak visible
\item The time resolution is about $\sim$ 1 ns
\end{itemize}
\end{beamercolorbox}


%\pause
%\vspace{.8cm}
%\hspace{1cm}\huge\color{tugreen}{Thank You :)}
\end{frame}
% \LogoOn





%\section{Backup}
%\begin{frame}
%\hspace{0.43\columnwidth}Backup
%\end{frame}
%




\end{document}