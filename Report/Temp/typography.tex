\section{Typography}
\label{sec:typography}

The use of the Latex typesetting symbols defined in the file
\texttt{lhcb-symbols-def.tex} and detailed in the appendices of this
document is strongly encouraged as it will make it much easier to
follow the recommendation set out below.

\begin{enumerate}

\item \lhcb is typeset with a normal (roman) lowercase b.

\item Titles are in bold face, and usually only the first word is
  capitalised.

\item Mathematical symbols and particle names should also be typeset
  in bold when appearing in titles.

\item Units are in roman type, except for constants such as $c$ or $h$
  that are italic: \gev, \gevcc.  The unit should be separated from
  the value with a thin space (``\verb!\,!''),
  and they should not be broken over two lines.
  Correct spacing is automatic when using predefined units inside math mode: \verb!$3.0\gev$! $\to 3.0\gev$.
  Spacing goes wrong when using predefined units outside math mode AND forcing extra space:
  \verb!3.0\,\gev! $\to$ 3.0\,\gev or worse:   \verb!3.0~\gev! $\to$ 3.0~\gev. 

\item  If factors of $c$ are kept, they should be used both for masses and
  momenta, \eg $p=5.2\gevc$ (or $\gev c^{-1}$), $m = 3.1\gevcc$ (or $\gev c^{-2}$). If they are dropped this
  should be done consistently throughout, and a note should be added
  at the first instance to indicate that units are taken with $c=1$.

\item The \% sign should not be separated from the number that precedes it: 5\%, not 5 \%. 
A thin space is also acceptable: 5\,\%, but should be applied consistently throughout the paper.

\item Ranges should be formatted consistently. The recommendend form is to use a dash with no spacing around it: 
7--8\gev, obtained as \verb!7--8\gev!. 

\item Italic is preferred for particle names (although roman is
  acceptable, if applied consistently throughout).  Particle Data
  Group conventions should generally be followed: \Bd (no need for a
  ``d'' subscript), \decay{\Bs}{\jpsi\phi}, \Bsb,
  (note the long bar, obtained with \verb!\overline!, in contrast to the discouraged short \verb!\bar{B}! resulting in $\bar{B}$), \KS (note the
  uppercase roman type ``S''). 
This is most easily achieved by using the predefined symbols described in 
  Appendix~\ref{sec:listofsymbols}.
  Unless there is a good reason not to, the charge of a particle should be
  specified if there is any possible ambiguity 
  ($m(\Kp\Km)$ instead of $m(KK)$, which could refer to neutral kaons).

\item Decay chains can be written in several ways, depending on the complexity and the number of times it occurs. Unless there is a good reason not to, usage of a particular type should be consistent within the paper.
Examples are: 
$\Dsp\to\phi\pip$, with $\phi\to\Kp\Km$; 
$\Dsp\to\phi\pip$ ($\phi\to\Kp\Km$);  
$\Dsp\to\phi(\to\Kp\Km)\pip$; or
$\Dsp\to[\Kp\Km]_\phi\pip$.



\item Variables are usually italic: $V$ is a voltage (variable), while
  1 V is a volt (unit). Also in combined expressions: $Q$-value, $z$-scale, $R$-parity \etc

\item Subscripts and superscripts are roman type when they refer to a word (such as T
  for transverse) and italic when they refer to a variable (such as
  $t$ for time): \pt, \dms, $t_{\mathrm{rec}}$.

\item Standard function names are in roman type: \eg $\cos$, $\sin$
  and $\exp$.

\item Figure, Section, Equation, Chapter and Reference should be
  abbreviated as Fig., Sect. (or alternatively Sec.), Eq., Chap.\ and
  Ref.\ respectively, when they refer to a particular (numbered) item,
  except when they start a sentence. Table and Appendix are not
  abbreviated.  The plural form of abbreviation keeps the point after
  the s, \eg Figs.~1 and~2. Equations may be referred to either with 
  (``Eq.~(1)'') or without (``Eq.~1'') parentheses, 
  but it should be consistent within the paper.

\item Common abbreviations derived from Latin such as ``for example''
  (\eg), ``in other words'' (\ie), ``and so forth'' (\etc), ``and
  others'' (\etal), ``versus'' (\vs) can be used, with the typography
  shown, but not excessively; other more esoteric abbreviations should be avoided.
  

\item Units, material and particle names are usually lower case if
  spelled out, but often capitalised if abbreviated: amps (A), gauss
  (G), lead (Pb), silicon (Si), kaon (\kaon), but proton (\proton).

%\item The prefix for 1000 (k, \eg kV) should not be confused with
%  that used in computing (K, which strictly speaking denotes $2^{10}$,
%  \eg KB).

\item Counting numbers are usually written in words if they start a
  sentence or if they have a value of ten or below in descriptive
  text (\ie not including figure numbers such as ``Fig.\ 4'', or
  values followed by a unit such as ``4\,cm'').
  The word 'unity' can be useful to express the special meaning of
  the number one in expressions such as: 
``The BDT output takes values between zero and unity''.
% Numbers should not be
%  written as words if they by nature are real numbers that happen to
%  take an integer value, such as $\chisq/\mathrm{ndf} < 4$.

\item Numbers larger than 9999 have a comma (or a small space, but not both) between
  the multiples of thousand: \eg 10,000 or 12,345,678.  The decimal
  point is indicated with a point rather than a comma: \eg 3.141.

\item We apply the rounding rules of the
  PDG~\cite{PDG2014}. The basic rule states that if the three
  highest order digits of the uncertainty lie between 100 and 354, we round
  to two significant digits. If they lie between 355 and 949, we round
  to one significant digit. Finally, if they lie between 950 and 999,
  we round up and keep two significant digits. In all cases,
  the central value is given with a precision that matches that of the
  uncertainty. So, for example, the result $0.827 \pm 0.119$ should be
  written as $0.83\pm 0.12$, $0.827\pm 0.367$ should turn into
  $0.8\pm 0.4$, and $14.674\pm0.964$ becomes $14.7\pm1.0$.
 When writing numbers with uncertainty components from
  different sources, \ie statistical and systematic uncertainties, the rule
  applies to the uncertainty with the best precision, so $0.827\pm
  0.367\stat\pm 0.179\syst$ goes to $0.83\pm 0.37\stat\pm 0.18\syst$ and
  $8.943\pm 0.123\stat\pm 0.995\syst$ goes to $8.94\pm 0.12\stat\pm
  1.00\syst$.

\item When rounding numbers, it should be avoided to pad with zeroes
  at the end. So $51237 \pm 4561$ should be rounded as $(5.12 \pm 0.46)
  \times 10^4$ and not $51200 \pm 4600$.

\item When rounding numbers in a table, some variation of the rounding
  rules above may be required to achieve uniformity.

\item Hyphenation should be used where necessary to avoid ambiguity,
  but not excessively. For example: ``big-toothed fish''
  (to indicate that big refers to the teeth, not to the fish),
  but ``big white fish''.
  A compound modifier often requires hyphenation 
  (\CP-violating observables, \bquark-hadron decays, final-state radiation, second-order polynomial),
  even if the same combination in an adjective-noun combination does not
  (direct \CP violation, heavy \bquark hadrons, charmless final state).
  Adverb-adjective combinations are not hyphenated if the adverb ends with 'ly':
  oppositely charged pions, kinematically similar decay.
  Cross-section, cross-check, and two-dimensional are hyphenated.
  Semileptonic, pseudorapidity, pseudoexperiment, multivariate, multidimensional, reweighted, preselection, 
  nonresonant, nonzero, nonparametric, nonrelativistic, misreconstructed and misidentified
  are single words and should not be hyphenated.

\item Minus signs should be in a proper font ($-1$), not just hyphens
  (-1); this applies to figure labels as well as the body of the text.
  In Latex, use math mode (between \verb!$$!'s) or make a dash (``\verb!--!'').
  In ROOT, use \verb!#font[122]{-}! to get a normal-sized minus sign. 

\item Inverted commas (around a title, for example) should be a
  matching set of left- and right-handed pairs: ``Title''. The use of
  these should be avoided where possible.

\item Single symbols are preferred for variables in equations, \eg\
  \BF\ rather than BF for a branching fraction.

\item Parentheses are not usually required around a value and its
  uncertainty, before the unit, unless there is possible ambiguity: so
  $\dms = 20 \pm 2\invps$ does not need parentheses, whereas $f_d =
  (40 \pm 4)$\% or $x=(1.7\pm0.3)\times 10^{-6}$ does.
  The unit does not need to be repeated in
  expressions like $1.2 < E < 2.4\gev$.

\item The same number of decimal places should be given for all values
  in any one expression (\eg $5.20 < m_B < 5.34\gevcc$).

\item Apostrophes are best avoided for abbreviations: if the abbreviated term
  is capitalised or otherwise easily identified then the plural can simply add
  an s, otherwise it is best to rephrase: \eg HPDs, \pizs, pions, rather
  than HPD's, \piz's, $\pion$s.

\item Particle labels, decay descriptors and mathematical functions are not nouns, and need often to be followed by a noun. 
Thus ``background from $\Bd\to\pi^+\pi^-$ decays'' instead of ``background from $\Bd\to\pi^+\pi^-$'',
and ``the width of the Gaussian function'' instead of ``the width of the Gaussian''.

\item In equations with multidimensional integrations or differentiations, the differential terms should be separated by a thin space. 
Thus $\int f(x,y) dx\,dy$ instead $\int f(x,y) dxdy$ and
$\frac{d^2\Gamma}{dx\,dQ^2}$ instead of $\frac{d^2\Gamma}{dxdQ^2}$.
The d's are allowed in either roman or italic font, but should be consistent throughout the paper.


\end{enumerate}
