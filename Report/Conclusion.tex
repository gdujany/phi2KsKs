\section{Conclusion}
The signal and background yields for the two approaches of $\phi$ selection have been calculated. The result suggests that the use of the topology $D_s^\pm \rightarrow \phi\pi^\pm$ does not offer significant improvement in the signal to background ratio.

The time resolution of the LHCb detector and the reconstruction algorithms commonly used by the collaboration have been studied in depth.

Finally, a toy study for the prompt $\phi$ approach has been conducted, resulting in the prediction that the luminosity needed at LHCb to reach the level of precision KLOE has for the measurement of CPT invariance in $\phi \rightarrow K^0\overline{K}^0$ is in the order of \SI{275}{fb^{-1}}.

It can be concluded that with the means currently available to LHCb, the study of CPT invariance in the given system does not seem feasible.